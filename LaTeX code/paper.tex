% ----- PREAMBLE -----

\documentclass[12pt,journal,compsoc]{IEEEtran}


\providecommand{\PSforPDF}[1]{#1}

\newcommand\MYhyperrefoptions{bookmarks=true,bookmarksnumbered=true,
pdfpagemode={UseOutlines},plainpages=false,pdfpagelabels=true,
colorlinks=true,linkcolor={black},citecolor={black},pagecolor={black},
urlcolor={black},
pdftitle={Title of the paper},% TODO: CHANGE!
pdfsubject={Typesetting},% TODO: CHANGE!
pdfauthor={Krystof Hruby},% TODO: CHANGE!
pdfkeywords={Computer Society, IEEEtran, journal, LaTeX, paper,
             template}}% TODO: CHANGE!

\hyphenation{op-tical net-works semi-conduc-tor} % TODO: CHANGE or remove



% ----- DOCUMENT -----

\begin{document}

\title{This is a title\\ for the paper}

\author{Krystof Hruby (W20021203)}

\markboth{KF5042 Intelligent Systems Module at Northumbria University Newcastle}{}


% --- Abstract --- 
\IEEEcompsoctitleabstractindextext{
  \begin{abstract}
    The abstract goes here.
  \end{abstract}

  \begin{IEEEkeywords}
    Computer Society, IEEEtran, journal, \LaTeX, paper, template.
  \end{IEEEkeywords}}


% --- Maketitle ---
\maketitle

\IEEEdisplaynotcompsoctitleabstractindextext

\IEEEpeerreviewmaketitle


% --- Introduction ---
\section{Introduction}

\IEEEPARstart{T}{his} demo file is intended to serve as a ``starter file''
for IEEE Computer Society journal papers produced under \LaTeX\ using
IEEEtran.cls version 1.7 and later.
% You must have at least 2 lines in the paragraph with the drop letter
% (should never be an issue)
I wish you the best of success.

\hfill mds

\hfill January 11, 2007

\subsection{Subsection Heading Here}
Subsection text here.

\subsubsection{Subsubsection Heading Here}
Subsubsection text here.


% --- Literature Review ---
\section{Literature Review}


% --- AI Experiments ---
\section{AI Experiments}


% --- Result Analysis ---
\section{Result Analysis}


% --- Conclusion and Future Work ---
\section{Conclusion and future work}
The conclusion goes here.

% --- Appendix
\appendices
\section{Proof of the First Zonklar Equation}
Appendix one text goes here.

% you can choose not to have a title for an appendix
% if you want by leaving the argument blank
\section{}
Appendix two text goes here.


% --- Acknowledgments --- TODO: probably remove
% use section* for acknowledgement
\ifCLASSOPTIONcompsoc
  % The Computer Society usually uses the plural form
  \section*{Acknowledgments}
\else
  % regular IEEE prefers the singular form
  \section*{Acknowledgment}
\fi
The authors would like to thank...


% --- References ---
% Can use something like this to put references on a page
% by themselves when using endfloat and the captionsoff option.
\ifCLASSOPTIONcaptionsoff
  \newpage
\fi

\begin{thebibliography}{1}

  \bibitem{IEEEhowto:kopka}
  H.~Kopka and P.~W. Daly, \emph{A Guide to {\LaTeX}}, 3rd~ed.\hskip 1em plus
  0.5em minus 0.4em\relax Harlow, England: Addison-Wesley, 1999.

  \bibitem{}
  This is a reference
\end{thebibliography}

% TODO: idk what this is for
\begin{IEEEbiography}{Michael Shell}
  Biography text here.
\end{IEEEbiography}

% if you will not have a photo at all:
\begin{IEEEbiographynophoto}{John Doe}
  Biography text here.
\end{IEEEbiographynophoto}

% insert where needed to balance the two columns on the last page with
% biographies
%\newpage
\begin{IEEEbiographynophoto}{Jane Doe}
  Biography text here.
\end{IEEEbiographynophoto}


\end{document}